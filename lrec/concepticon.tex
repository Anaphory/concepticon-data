\documentclass[10pt, a4paper]{article}
\clubpenalty1000000
\widowpenalty1000000
\usepackage{lrec2014}
\usepackage{graphicx}
\usepackage[pdfborder={0 0 0}]{hyperref}
\usepackage{listings}
\usepackage{amsmath,amssymb,colortbl,xcolor}
\usepackage{tipa}

\title{Concepticon: A Resource for the Linking of Concept Lists}

\name{Johann-Mattis List$^{1}$, Michael Cysouw$^{2}$, Robert Forkel$^{3}$}

\address{ 
               $^{1}$CRLAO/EHESS and AIRE/UPMC, Paris,
	       $^{2}$Forschungszentrum Deutscher Sprachatlas, Marburg,\\
	       $^{3}$Max Planck Institute for the Science of Human History, Jena
	       \\
	       \href{mailto:mattis.list@lingpy.org}{mattis.list@lingpy.org}, 
	       \href{mailto:email}{emailpending}, 
	       \href{mailto:email}{emailpending}\\
	       }


\abstract{
We present an attempt to link the large amount of different concept lists (aka ``Swadesh lists'')
which are used in the linguistic literature.  
\newline \Keywords{concepts, concept lists, Swadesh lists, cross-linguistically linked data}}


% this command serves to get easy access to the basis in hyperref
\newcommand{\concepticon}[1]{http://concepticon.clld.org/#1}
\newcommand{\conceptnumber}{\textcolor{red}{xxx}}
\newcommand{\listnumber}{\textcolor{red}{xxx}}
\newcommand{\setnumber}{\textcolor{red}{xxx}}
\newcommand{\relationnumber}{\textcolor{red}{xxx}}

\begin{document}

\maketitleabstract

\section{Introduction}
\textcolor{red}{Start with Swadesh and explain, how this whole idea started, namely, that concept
lsits are elicited}
\begin{quote}
[...] it is a well known fact that certain types of morphemes are relatively stable. Pronouns and
numerals, for example, are occasionally replaced either by other forms from the same language or by
borrowed elements, but such replacement is rare. The same is more or less true of other everyday
expressions connected with concepts and experiences common to all human groups or to the groups
living in a given part of the world during a given epoch. \cite[pagepending]{Swadesh1950}
\end{quote}


\begin{quote}
  \textcolor{red}{this collection of Swadesh's first concepts may be included here, but we may also
  consider just to have it put into some kind of a graphic}
  I, thou, he, we, ye, one, two, three, four, five,
six, seven, eight, nine, ten, hundred, all,
animal, ashes, back, bad, bark, belly, big,
black, blood, bone, brother (elder), child
(son or daughter), cloud, cold, come, cry
(weep), dance, day, dog, dust, ear, earth, eat,
egg, eye, far, father, fire, flower, fog, foot,
good, grass, green, hair, hand, head, heart,
here, hit (with fist), hunt, husband, ice,
lake, laugh, leaf, left hand, leg, liver, long,
louse, man, meat, mother, mountain, mouth,
name, near, neck, night, nose, person, rain,
red, right hand, road (trail), root, rope, salt,
sand, short, sing, sister (elder), skin, sky,
small, smoke, snake, snow, speak, spear
(war), star, stone, sun, swim, tail, that, there,
this, tongue, tooth, tree, warm, water, what,
where, white, who, wife, wind, woman, year,
yellow. \cite[161]{Swadesh1950}
\end{quote}

\section{Concept Lists}
\textcolor{red}{Here, we should work on some basic characteristics of concept lists, among these is
the basic definition we give to a concept list (as follows, from the slides), but also further
points}
Simply speaking, concept lists are lists of concepts, in which
concepts are ideally given by both glosses and short definitions. 
They can be compiled for different purposes (language
comparison, concept comparison) and be expanded by adding structure (rankings, divisions, relations).

\subsection{Purpose of Concept Lists}
\textcolor{red}{The purpose of concept lists should probably be mentioned her. Note that the purpose
is never completely exhaustively described, but we can distinguish between different basic types,
and this makes probably sense}
\begin{itemize}
  \item {\bf Language Comparison (historical linguistics, dialectology)}
    \begin{itemize}
      \item proving genetic relationship (Yakhontov 1991/35 items, Dolgopolsky 1964/15 items)
      \item linguistic subgrouping (Norman 2003/40 items, Swadesh 1955/100 items, Starostin 1991/110 items)
      \item layer identification (Chén 1996/100+100 items, Yakhontov 1991/35+65 items)
    \end{itemize}
  \item Concept comparison (historical lingusitics, psycholinguistics)
    \begin{itemize}
      \item synchronic (word association: SimLex, Hill et al. 2014/1028 items,
	colexification: CLICS, List et al. 2014/1280 items)
      \item diachronic (semantic shift: DatSemShift, Bulakh et al. 2013/2424
	items, stability of form-meaning relations: WOLD, Haspelmath \&
	Tadmor 2009/1460 items)
    \end{itemize}
\end{itemize}

\subsection{Structure of Concept Lists}
\textcolor{red}{The structure, and the qeustion as to what the swadesh lists are aactually intended
to do should probably briefly be mentioned here. Using the table as it was prepared for the talk in
Leipzig may be useful, as I think, but we can probably float it and have it cover two columns}
\resizebox{0.5\textwidth}{!}{%
\tabular{|p{4cm}|p{4cm}|p{4cm}|}
\hline
\bf Type & \bf Example & \bf Purpose \\\hline\hline 
basic vocabulary list (``Swadesh list") & Swadesh 1952 / 200 items & subgrouping \\\hline 
subdivided concept list & Yakhontov 1991 / 35 + 65 items & genetic relationship, layer identification \\\hline
``ultra-stable" concept list & Dolgopolsky 1964 / 15 items & genetic relationship \\\hline
questionnaire & Allen 2007 / 500 items & dialect / language comparison \\\hline
ranked list & Starostin 2007 / 110 items & subgrouping, layer identification \\\hline
list of concept relations & DatSemShift, Bulakh et al. 2013 / 2424 items & representation of concept
relations \\\hline
special-purpose concept list & Matisoff 1978 / 200 items & subgrouping of Tibeto-Burman languages \\\hline
historical concept list & Leibniz 1768 / 128 items & language comparison \\\hline
\endtabular}

\section{Linking Concept Lists}
\textcolor{red}{Here, we should describe the basic characteristics of the concepticon, like the way
the things are linked with each other. Maybe, including a graphic would also be useful}
The concepticon is an attempt to link the many different concept lists (``Swadesh Lists'') which are
used in the linguistic literature. In practice, all entries from the various concept lists are
linked to a \emph{concept set} as an intermediate way to reference the concepts. The Concepticon
currently links \conceptnumber concepts from \listnumber concept lists to \setnumber concept sets
and defines \relationnumber relations between the concept sets.
 
A concept list is a collection of concepts that is deemed interesting by scholars. Minimally, it
consists of an \emph{identifier} for each concept which the lists contains, and a \emph{label} by
which the concept is referenced. The creator of a concept list is called a \emph{compiler}. Each
concept list is tight to one or more \emph{sources}, it is given in one or more \emph{source
languages} and was compiled for one or more \emph{target languages}. A \emph{description} gives
further information on each concept list in free, exclusively human-readable form.
 
To facilitate our workflow and to guarantee the comparability of concept lists even if they do not
share concepts which are directly linked via our concept sets, we define additional and very simple
\emph{concept relations} between concept sets (\emph{broader}, \emph{narrower}, \emph{similar}).
Even if the concepts in two or more concept lists are not assigned to the same concept set, they can
still be assigned to concept sets via concept relations.

\section{Examples}
\textcolor{red}{Examples may be useful to be included, they could, however, also be put into a nice
graphic in which the web-application is presented along with the underlying graphs showing the
relations between the concepts}
\subsection{Rain}

\textcolor{red}{Maybe give rain as an example here, as in the slides}

\subsection{Child}

\textcolor{red}{Child example, for hierarchies}

\subsection{Burn}
\textcolor{red}{Burn example to show problems with transitivity etc.}

\section{Using the Concepticon}
\textcolor{red}{HIer eventuell zeigen, wie das Concepticon bei Dictionaria und Lexibank benutzt
werden kann}

\section{Outlook}
\textcolor{red}{heir noch mal sagen, dass wir natürlich noch weiter daran arbeiten.}


\section{Acknowledgements}
\textcolor{red}{vielleicht nicht nötig im Moment...}

\bibliographystyle{lrec2014}
\bibliography{concepticon}

\end{document}

